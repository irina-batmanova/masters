\documentclass[16pt]{article}
\usepackage[utf8]{inputenc}
\usepackage[T2A]{fontenc}
\usepackage[utf8]{inputenc}
\usepackage[russian]{babel}
\usepackage{ amssymb }
\usepackage{mathtools}


\title{Проверка эквивалентности CDP}

\begin{document}

\maketitle

\section{Алгоритм проверки эквивалентности}
Считаем, что трансформация базы всегда происходит только 1 раз, и это первая операция в цепочке преобразований (ниже будет доказано, почему такое предположение корректно)
\\

\begin{itemize}
\item[1] Выберем нумерацию вершин в $\square'$
\item[2] Найдем такую матрицу $A$, которая переводит $\square$ в $\square'$ (с учетом нумерации вершин), найдем матрицу обратного преобразования - $A^{-1}$, применим это преобразование к $\psi_1, ..., \psi_n$, получим $\psi_1\phi^{-1}, ..., \psi_n\phi^{-1}$
\item[3] Проверим, что $\sum_{i=1}^n\psi_i\phi^{-1}(u_{new}) = \sum_{i=1}^n\psi_i^{'}(u_{new})$ для всех вершин областей линейности функций (исходные $\psi_i$ кусочно-линейны, и области линейности - многогранники). Если это верно для всех указанных $u$ - CDP эквивалентны, завершаем алгоритм. Если нет - повторяем шаги 1-3 для другой нумерации вершин
\item[4] Если дошли до этого шага - CDP не эквивалентны, завершаем алгоритм

\end{itemize}

\section{Доказательство корректности алгоритма}
\subsection{Почему преобразование базы можно всегда считать первым в цепочке преобразований?}
Пусть $CDP_1\ (\square, \psi_1, ..., \psi_n)$ переводится в $CDP_2\ (\square^{'}, \psi_1^{'}, ..., \psi_n^{'})$ с помощью цепочки преобразований $BC$, где $B$ - операция сдвига, $C$ - трансформация базы. Покажем, что тогда существует другая цепочка преобразований - $C^{'}B^{'}$, где  $C^{'}$ - трансформация базы, $B^{'}$ - операция сдвига, - которая также переводит $CDP_1$ в $CDP_2$.
\\
$\psi_i(u) \xrightarrow[]{C^{'}B^{'}} \psi_i^{'}(u_{new}) = \psi_i(\phi^{-1}(u_{new})) + \beta_i<u_new, v>$
\\
$\psi_i(u) \xrightarrow[]{B} \psi_i(u) + \beta_i<u, v_1>$, положим $v_1 = \phi(v)$, тогда $\psi_i(u) \xrightarrow[]{B} \psi_i(u) + \beta_i<u, \phi(v)>$, теперь применим трансформацию $C$: 
$\psi_i(u) \xrightarrow[]{BC} \psi_i(\phi^{-1}(u_{new})) + \beta_i<\phi^{-1}(u_{new}), \phi(v)> = \psi_i(\phi^{-1}(u_{new})) + \beta_i<u_{new}, v> = \psi_i^{'}(u_{new})$, то есть цепочки преобразований $BC$ и $C^{'}B^{'}$ дают одинаковый результат

Аналогично для операции перемещения (translation)

\subsection{Почему из равенства сумм $\psi_i$ и $\psi_i^{'}$ на всех  вершинах областей линейности следует существование цепочки преобразований, переводящих набор $\psi_i$ в набор $\psi_i^{'}$}
По предыдущему пункту доказательства корректности можно считать, что на данном шаге алгоритма функции уже заданы на одинаковых многогранниках (трансформация базы была первой в цепочке преобразований, а все другие преобразования не меняют базу). Сумма кусочно-линейных функций кусочно-линейна (и просто линейна многогранниках, заданных пересечением всех многогранников-областей линейности слагаемых) - значит, из равенства сумм $\psi_i$ и $\psi_i^{'}$ на всех  вершинах областей линейности следует равенство сумм на всех точках $\square$
Докажем, что из равенства сумм на всех точках базы следует существование цепочки преобразований одного набора функций в другой набор.

\end{document}